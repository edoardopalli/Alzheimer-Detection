\documentclass{beamer}
%pacchetti
\usepackage[T1]{fontenc}
\usepackage[utf8]{inputenc}
\usepackage{graphicx}
\usepackage[italian]{babel}
\usepackage{mathrsfs}
\usepackage{booktabs}
\usepackage{amsmath}
\usepackage{amsfonts}
\usepackage{amssymb}
\usepackage{amsbsy}
\usepackage{amsthm}
\usepackage{enumerate}
\usepackage{quoting}
\quotingsetup{font=small}
\usepackage{diagbox}
\usepackage{graphicx}
\usepackage{setspace}
\usepackage{float}
\usepackage{version}
\usepackage{multicol}
\usepackage{beamerfoils}

\usepackage[none]{hyphenat} %avoid hyphenation
\usepackage{xcolor} %to uset \textcolor
% end pacchetti

\usetheme[bgphoto]{polimi}

% Full instructions available at:
% https://github.com/elauksap/beamerthemepolimi

% Set custom font (requires to compile with XeLaTeX).
\usepackage{ifxetex}
\ifxetex
\usepackage{fontspec}
\setsansfont[Scale=0.8]{Arial}
\fi

\usepackage{lipsum}


\newcommand\mynum[1]{%
	\usebeamercolor{enumerate item}%
	\tikzset{beameritem/.style={circle,inner sep=0,minimum size=2ex,text=enumerate item.bg,fill=enumerate item.fg,font=\footnotesize}}%
	\tikz[baseline=(n.base)]\node(n)[beameritem]{#1};%
}


\title{Is Dementia predictable?}

%\subtitle{Subtitle}
\author{F. Di Filippo, E. Manfrin, E. Musiari, E. Palli}
\date{17 december 2021}



\begin{document}

\begin{frame}
\maketitle
\end{frame}


\begin{frame}{Dataset}

Dataset Dementia and Alzheimer longitudinal
%\vspace{0.2 cm}
\begin{center}
	
	
	\includegraphics[width=\columnwidth]{dataset_al.jpeg}
\end{center}


where SES is Socioeconomic Status, MMSE is Mini Mental State Examination, CDR is Clinical Dementia Rating, eTIV is Estimated Total Intracranial Volume, nWBV is Normalize Whole Brain Volume and ASF is Atlas Scaling Factor.

\vspace{0.1 cm}
Source: Kaggle


\end{frame}

\begin{frame}{Problem of Correlation}
We tried to solve the problem of correlation using the PCA method.

\end{frame}

\begin{frame}{Analysis Male VS Female}
% tra demented e nondem
Separating Demented and Nondemented
	
\end{frame}


\begin{frame}{Demented VS Nondemented (1)}
% visita 1 vs 2
Male

\end{frame}

\begin{frame}{Demented VS Nondemented (2)}
% visita 1 vs 2
Female

\end{frame}

\begin{frame}{Permutation}
% dem vs nondem

\end{frame}

\begin{frame}{2-ways manova}
% gruppi: m f dem nondem

\end{frame}

\begin{frame}{regression}

% modello logistico e spiegare: demented and non demented e le covariate usate: ses, normal brain volume, age, mmse -> smoothing splines; cdr senza smoothing perchè pochi valori distinti
% fittato su tutto il dataset esclusi i converted

% summary, shapiro e qqplot



\end{frame}

\begin{frame}{regression}

% prediction sui converted

% prediction su alcune persone non demented (tolte dal train del modello)

\end{frame}

\begin{frame}{Prediction}

\end{frame}

\begin{frame}{Survival Analysis}

\vspace{0.3 cm}
Event: disease occurred

	\begin{center}
		\includegraphics[width=0.9\columnwidth]{survival_plot2.jpeg}
	\end{center}
\end{frame}

\begin{frame}{Future questions}

\end{frame}


\end{document}